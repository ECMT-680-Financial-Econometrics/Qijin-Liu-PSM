\documentclass[final]{beamer}
\usepackage[size=a0,scale=1.4]{beamerposter}
\usetheme{Berlin}
\usecolortheme{rose}
\usepackage{graphicx, amssymb, epstopdf, setspace, url, natbib, semtrans, comment, pdfpages, enumerate, float, tocloft, caption, fancyhdr, rotating, lscape, pdflscape, lipsum, pbox, verbatim, multirow, threeparttable, colortbl, chngcntr, ragged2e, booktabs, tabularx, amsmath, tabularray, longtable, threeparttablex, threeparttable, subcaption, bbm}
\usepackage[english]{babel}
\usepackage[font=small,labelfont=bf]{caption}
\usepackage{tikz, tabularray, appendix, etoolbox, textcase}
\renewcommand{\thetable}{\arabic{table}}
\setcounter{table}{0}
\renewcommand{\thefigure}{\arabic{figure}}
\setcounter{figure}{0}

\title{\Huge Heat Exposure and Youth Migration in Central America and the Caribbean}
\author{\Large Javier Baez, German Caruso, Valerie Mueller, Chiyu Niu}
\institute{\Large World Bank, International Food Policy Research Institute, University of Illinois}
\date{\Large\today}

\setbeamertemplate{footline}{} 

\begin{document}
\begin{frame}[t]

\begin{block}{}
\centering
\maketitle
\end{block}

\begin{columns}[T]

\begin{column}{.32\textwidth}
    \begin{block}{\Huge Abstract}
    \Large
    This study assesses the impact of prolonged heat exposure on youth migration patterns in Central America and the Caribbean, utilizing a triple difference-in-difference analytical framework to analyze inter-province migration.
    \end{block}

    \vspace{1cm}

    \begin{block}{\Huge Introduction}
    \Large
    We explore the role of environmental changes, particularly heat exposure, in driving migration decisions among the youth in Central America and the Caribbean, a region with scarce literature on environmental drivers of migration.
    \end{block}

    \vspace{1cm}

    \begin{block}{\Huge Literature Review}
    \Large
    Previous studies have highlighted temporary migration in response to localized shocks, but the impact of prolonged environmental changes on permanent migration remains less understood.
    \end{block}
\end{column}

\begin{column}{.32\textwidth}
    \begin{block}{\Huge Methodology}
    \Large
    Employing a quasi-experimental triple DID design, we analyze census data across several countries in the region, focusing on temperature extremes' impact on migration, stratified by age groups to understand mobility patterns.
    \end{block}

    \vspace{1cm}

    \begin{block}{\Huge Findings}
    \Large
    Our findings suggest that heat exposure significantly influences migration, with young individuals and women more likely to migrate, particularly to provincial capitals. The migration response is more pronounced among the unskilled population.
    \end{block}
\end{column}

\begin{column}{.32\textwidth}
    \begin{block}{\Huge Discussion}
    \Large
    The analysis reveals complex interplay between environmental factors and migration decisions, highlighting the necessity of considering environmental variables in migration dynamics.
    \end{block}

    \vspace{1cm}

    \begin{block}{\Huge Conclusions}
    \Large
    This study contributes to understanding the environmental drivers of migration, emphasizing the need for policy interventions to minimize distress migration and manage the impacts of climate change.
    \end{block}

    \vspace{1cm}

    \begin{block}{\Huge References}
    \Large
    A detailed list of references supporting this analysis.
    \end{block}

    \vspace{1cm}

    \begin{block}{\Huge Appendix}
    \Large
    Supplementary material including detailed statistical analyses.
    \end{block}
\end{column}

\end{columns}
\vspace{1cm}

\begin{beamercolorbox}[center]{section in head/foot}
\Large Visit the project page or contact the authors for a downloadable version of this study and additional resources.
\end{beamercolorbox}

\end{frame}
\end{document}
